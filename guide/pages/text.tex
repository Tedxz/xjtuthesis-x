% !TEX root = ../guide.tex
% text.tex
%
% Copyright 2016 Zheng Xie <xie.zheng777@gmail.com>
% https://github.com/Tedxz/xjtuthesis-x
%
% This work may be distributed and/or modified under the
% conditions of the LaTeX Project Public License, either version 1.3
% of this license or (at your option) any later version.
% The latest version of this license is in
%   http://www.latex-project.org/lppl.txt
% and version 1.3 or later is part of all distributions of LaTeX
% version 2005/12/01 or later.
%
% This work has the LPPL maintenance status `maintained'.
%
% The Current Maintainer of this work is Zheng Xie.
%
% xjtuthesis-x is a Derived Work of xjtuthesis. The original maintainer of
% xjtuthesis is Weisi Dai (multiple1902 <multiple1902@gmail.com>),
% who published the project on https://code.google.com/p/xjtuthesis/ (no
% longer accessable). Currently, xjtuthesis is maintained by Aetf, and can
% be accessed on https://github.com/Aetf/xjtuthesis.
%
% xjtuthesis-x includes bug fixes, new features and a user guide.
% For detail, please refer to Readme.md.
%
% If you want to contribute to xjtuthesis-x or become the maintainer of
% xjtuthesis-x, please feel free to contact me.

\chapter{排版文字内容}
文字排版是\LaTeX 的基础内容,本章仅作简单介绍。若希望详细了解本章所设计到的内容,可以查阅任何的\LaTeX 入门资料。

\section{标题}
可以使用\verb|\chapter{章标题}|命令新建一章;使用\verb|\section{节标题}|命令新建一节;使用\verb|\subsection{小节标题}|命令新建小节。如果需要,可以使用\verb|\echapter{Chapter Title}|等命令添加英文标题。

\section{列表结构}
列表结构包括非标号列表(\texttt{itemize})、标号列表(\texttt{enumerate})和描述列表(\texttt{description}),是\LaTeX 中的常用功能。这里不做详细介绍,如需了解可以参考https://en.wikibooks.org/wiki/LaTeX/List\_Structures 。模板中使用了\texttt{enumitem}宏包提供了对列表间距的控制,如果需要调整列表项目间距或与前后正文的间距,可以参考该宏包的文档。

\begin{figure}[h]
  {
  \setstretch{1.2}%
  \fontsize{10pt}{12pt}\selectfont
  \setmainfont{Courier New}
  \begin{lstlisting}[showstringspaces=false,numbers=left,xleftmargin=3em]
\begin{enumerate}
\item The first item
\item The second item
\item The third etc \ldots
\end{enumerate}

  \end{lstlisting}
  }
\caption{图片插入代码样例}
\label{fig:fig_code}
\end{figure}

\section{公式排版}
使用\texttt{xjtuthesis-x}时插入公式只需要按照\LaTeX 的常用方法插入即可;模板可以按照正确的方式进行标号。根据具体需求,可以混合使用\texttt{equation}环境、\texttt{gather}环境和\texttt{align}环境。对于向量等加粗变量,建议使用\texttt{bm}宏包提供的\texttt{bm}命令,如\verb|$\bm{x}$|。

\section{参考文献}
生成参考文献列表需要在工作目录下建立\texttt{bibliography.bib},并将参考文献的\textsc{Bib}\TeX 信息加入其中。\verb|\xjtubib{bibliography}|命令可以生成符合规范的参考文献列表。在正文中引用文献时,使用\verb|cite|命令即可。
