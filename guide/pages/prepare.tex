% !TEX root = ../guide.tex
% prepare.tex
%
% Copyright 2016 Zheng Xie <xie.zheng777@gmail.com>
% https://github.com/Tedxz/xjtuthesis-x
%
% This work may be distributed and/or modified under the
% conditions of the LaTeX Project Public License, either version 1.3
% of this license or (at your option) any later version.
% The latest version of this license is in
%   http://www.latex-project.org/lppl.txt
% and version 1.3 or later is part of all distributions of LaTeX
% version 2005/12/01 or later.
%
% This work has the LPPL maintenance status `maintained'.
%
% The Current Maintainer of this work is Zheng Xie.
%
% xjtuthesis-x is a Derived Work of xjtuthesis. The original maintainer of
% xjtuthesis is Weisi Dai (multiple1902 <multiple1902@gmail.com>),
% who published the project on https://code.google.com/p/xjtuthesis/ (no
% longer accessable). Currently, xjtuthesis is maintained by Aetf, and can
% be accessed on https://github.com/Aetf/xjtuthesis.
%
% xjtuthesis-x includes bug fixes, new features and a user guide.
% For detail, please refer to Readme.md.
%
% If you want to contribute to xjtuthesis-x or become the maintainer of
% xjtuthesis-x, please feel free to contact me.

\chapter{准备工作与上手}

\section{毕业论文能用\LaTeX 写吗?}
毕业论文能否使用\LaTeX 完成,学校是否会接受往往是考虑使用\LaTeX 撰写毕业论文首先面临的问题。根据经验,学校的论文系统可以接受PDF作为论文的文件格式用以存档和查重,以往使用\LaTeX 撰写毕业论文的学长们也都顺利完成了答辩\footnote{据笔者所知,有少数计算机相关专业和许多数学专业的本科生使用\LaTeX 撰写毕业论文。};因此你可以放心地使用\LaTeX 撰写毕业论文。但是,需要注意以下几个问题:
\begin{enumerate}
  \item 提前与导师沟通。导师可能为了便于查看或修改,向你索要Word版本的论文。提前与导师沟通并获得认可是必要的。
  \item 第三方查重。在PaperPass等第三方网站进行查重时,由于缺少Word版本,可能需要手动将论文内容复制出来。
  \item 校级优秀论文缩写稿。申报校级优秀论文需要提交Word版本的缩写稿,但是要在极短的时间内(可能是一天左右的时间,甚至更短)缩写并用Word重新排版可能难于登天,许多\LaTeX 生成的格式很难在Word中复现。如果提前有申请评优的打算,可以提前准备缩写稿。\footnote{笔者的缩写稿就在慌忙之中完成,复杂的表格只得截图插入正文。草率的排版难免贻笑大方,但限于提交时间紧迫也无可奈何。}
\end{enumerate}

好在以上问题都不是难以克服的障碍。至于使用\LaTeX 撰写毕业论文的好处,对于考虑使用\LaTeX 的本文读者来说,想必无需多言。

\section{\texttt{xjtuthesis}还是\texttt{xjtuthesis-x}?}

\texttt{xjtuthesis-x}基于\texttt{xjtuthesis}添加了部分功能并做了修改。改动主要有:
\begin{enumerate}
  \item 添加了伪代码段环境和可跨页的为代码段环境(实验性);
  \item 调整了表格中的默认行距,更加美观;
  \item 调整了脚注行距;
  \item 将脚注标号与分割线对齐;
  \item 调整了行距参数,设置为1.2或1.5倍行距时,分别与官方Word模板中相同;
  \item 按照论文规范,移除段间距(实验性);
  \item 表格前空0.5行;
  \item 将公式字体设置为Computer Modern;
  \item 修正页眉的位置;
  \item 修正文字与页眉的距离。
\end{enumerate}
其中,最后两条修复已经合并至\texttt{xjtuthesis}项目中,其他改动则为\texttt{xjtuthesis-x}所特有。这些改动中,在《工作手册》中明确规范的条目后期可能会合并到\texttt{xjtuthesis}中,实验性功能或规范中未作规定的特性则将保持\texttt{xjtuthesis-x}独有。

\texttt{xjtuthesis-x}和\texttt{xjtuthesis}中,关键的模板文件\texttt{xjtuthesis.cls}可以相互代替。因此,可以先选择任一模板开始论文写作并随时切换。

\section{编译第一个版本}

首先将根目录中的\texttt{xjtuthesis.cls}和\texttt{gbt7714-2005-xjtu.bst}两个文件拷贝至\texttt{framework/}目录下。\texttt{framework/}目录包含了可以用于开始写作的文件结构。如果你喜欢,可以删除这个目录以外的所有文件,并把\texttt{framework}改成任意名字后开始写作。\texttt{meta.tex}中的信息需要根据注释填写完整。
Windows用户可以在\texttt{bachelor.tex}、\texttt{master.tex}和\texttt{doctor.tex}中选择一个要用的改名为\texttt{thesis.tex},删除另外两个;然后使用\texttt{build.bat}编译。Linux/OSX用户可以使用\texttt{make}命令。此外,还可以使用任何你喜欢使用的TeX编辑器提供的编译功能。

一个好的习惯是把文章按照章节或更细的粒度分为不同的文件,并使用\texttt{input}命令加入根文件中。在\texttt{pages/}目录下可以看到一些已经分好的章节,读者可以根据自己的需求创建文件进行写作。为了方便调试,笔者甚至建立了\texttt{tables/}目录,并将所有表格的\texttt{tabular}环境中的内容作为单独的文件放在里面。
