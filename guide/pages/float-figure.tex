% !TEX root = ../guide.tex
% float-figure.tex
%
% Copyright 2016 Zheng Xie <xie.zheng777@gmail.com>
% https://github.com/Tedxz/xjtuthesis-x
%
% This work may be distributed and/or modified under the
% conditions of the LaTeX Project Public License, either version 1.3
% of this license or (at your option) any later version.
% The latest version of this license is in
%   http://www.latex-project.org/lppl.txt
% and version 1.3 or later is part of all distributions of LaTeX
% version 2005/12/01 or later.
%
% This work has the LPPL maintenance status `maintained'.
%
% The Current Maintainer of this work is Zheng Xie.
%
% xjtuthesis-x is a Derived Work of xjtuthesis. The original maintainer of
% xjtuthesis is Weisi Dai (multiple1902 <multiple1902@gmail.com>),
% who published the project on https://code.google.com/p/xjtuthesis/ (no
% longer accessable). Currently, xjtuthesis is maintained by Aetf, and can
% be accessed on https://github.com/Aetf/xjtuthesis.
%
% xjtuthesis-x includes bug fixes, new features and a user guide.
% For detail, please refer to Readme.md.
%
% If you want to contribute to xjtuthesis-x or become the maintainer of
% xjtuthesis-x, please feel free to contact me.

\section{图片的插入}

一个简单的图片插入的代码如图~\ref{fig:fig_code}所示。若要引用该图片,在正文中插入第一行的内容即可。``图''字后面的波浪线产生一个不可间断空格,使得在编译时不会在该处换行,后通过\texttt{ref}命令引用图片标号,得到类似``图~1-3''的效果。代码中,\texttt{figure}环境创建了一个浮动的块,在该块中\texttt{includegraphics}命令插入了具体的图片\footnote{事实上,\texttt{figure}环境中并不一定要插入图片,如图~\ref{fig:fig_code},里面实际上是文本。}。
所有的图片放在\texttt{figures/}目录下,这是通过在最上层的\texttt{tex}中,\texttt{\\graphicspath\string{\string{figures/\string}\string}}这行代码设定的。参数\texttt{h}表示图片尽量防止在代码的位置,如果放不下的话可以浮动\footnote{参数设置参见\url{https://en.wikibooks.org/wiki/LaTeX/Floats,_Figures_and_Captions}。}。

\begin{figure}[h]
  {
  \setstretch{1.2}%
  \fontsize{10pt}{12pt}\selectfont
  \setmainfont{Courier New}
  \begin{lstlisting}[showstringspaces=false,numbers=left,xleftmargin=3em]
图~\ref{fig:figlabel}

\begin{figure}[h]
  \centering
  \includegraphics[width=.5\textwidth]{picture.eps}
  \caption{图注内容}
  \label{fig:figlabel}
\end{figure}

  \end{lstlisting}
  }
\caption{图片插入代码样例}
\label{fig:fig_code}
\end{figure}

\texttt{xjtuthesis-x}中使用\texttt{subfig}包实现子图。如果要插入包含多个子图的图片,并对每一个子图进行标注,可以参考图~\ref{fig:fig_demo_subfig}的代码。

\begin{figure}[h]
  {
  \setstretch{1.2}%
  \fontsize{10pt}{12pt}\selectfont
  \setmainfont{Courier New}
  \begin{lstlisting}[showstringspaces=false,numbers=left,xleftmargin=3em]
\begin{figure}[h]
  \centering
  \subfloat[][第一个图的图注]{
    \includegraphics[width=0.32\textwidth]{fig1_1}
  }
  \subfloat[][第二个图的图注]{
    \includegraphics[width=0.32\textwidth]{fig1_2}
  }
  \subfloat[][第三个图的图注]{
    \includegraphics[width=0.32\textwidth]{fig1_3}
  }
  \caption{总体的图注}
  \label{fig:subfig}
\end{figure}


  \end{lstlisting}
  }
\caption{子图插入代码样例}
\label{fig:fig_demo_subfig}
\end{figure}

值得一提的是,\LaTeX 中默认使用的是eps格式的矢量图。模板中使用的\texttt{graphicx}包使得文中可以插入jpg等格式的图片。在论文中使用矢量图可以使得论文图片看起来更清晰,因此强烈建议大家使用矢量图。Matlab等许多软件可以生成eps格式的矢量图;也可以用Adobe Illustrator绘制矢量图。
